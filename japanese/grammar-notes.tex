\documentclass[a4paper, 11pt]{book}

\usepackage[main=spanish,japanese]{babel}
\usepackage[utf8]{inputenc}
\usepackage{CJKutf8} % Para poder escribir caracteres CJK
\usepackage{tcolorbox} % Para la caja de explicacion rapida
\usepackage{graphicx} % Para imagenes
\usepackage{hyperref} % Para permitir que el indice sea clicable.

\setcounter{tocdepth}{4}

\definecolor{celeste}{HTML}{E0F7FA}

\title{Apuntes de gramática}
\author{Nombre}

%% ----------------------------------------------
% Comando para insertar texto en japones en cualquier
% momento que desee el escritor.
%
% Parametros:
% 	* El unico argumento es el texto en japones a mostrar.
%% ----------------------------------------------
\newcommand{\tjp}[1]{\begin{CJK}{UTF8}{min}#1\end{CJK}}

%% ----------------------------------------------
% Comando para insertar un ejemplo en japones.
% Se constituye de dos argumentos, la parte en japones
% y su traducción al español.
%
% Parametros:
% 	* El primero argumento es el texto japones del ejemplo.
% 	* El segundo argumento es su tradución al español.
%% ----------------------------------------------
\newcommand{\jpex}[2] {
	\vspace{1.0\baselineskip}
	\begin{CJK}{UTF8}{min}
		\noindent\textbf{\large{#1}}
		$\rightarrow$ \emph{#2}
	\end{CJK}
}

%% ----------------------------------------------
% Comando para indicar las formas gramaticales
% que preceden a cierta forma gramatical
%
% Parametros:
% 	* El primero argumento es las formas gramaticales que preceden
%		a la forma gramatical tratada.
% 	* El segundo argumento es la forma gramatical en cuestion.
%% ----------------------------------------------
\newcommand{\grpre}[2]{
	\begin{center}
		\begin{CJK}{UTF8}{min}
			\begin{tabular}{l}
				#1
			\end{tabular}
			$+$
			\begin{tabular}{l}
				#2
			\end{tabular}
		\end{CJK}
	\end{center}
}

%% ----------------------------------------------
% Seccion donde aparece la deficion rapida de una expresion
% gramatical. 
%
% Parametros:
% 	* Argumento opcional entre [] que servira de titulo de la caja.
%% ----------------------------------------------
\newenvironment{defrapida}[1][\unskip]{
	\noindent
	\begin{CJK}{UTF8}{min}
		\begin{tcolorbox}[colback=celeste, title=#1]
} {
		\end{tcolorbox}
	\end{CJK}
}

\begin{document}
	\maketitle
	\tableofcontents
	\listoftables
	\listoffigures

	%%%%%%%%%%%%%%%%%%%%%%%%%%%%
	%% Contenido de ejemplo eliminar para el uso.	%
	%%%%%%%%%%%%%%%%%%%%%%%%%%%%

	% Importante usar protect para que se definan correctamente en el .toc
	% los titulos usando caracteres CJK. Si no se usa protect cuando se usa tjp
	% en un titulo que aparece en el indice los que vengan a continuacion no
	% funcionaran.
	\chapter{\protect\tjp{始めましょう!}}
	\section{Ejemplo sección}
	\subsection{Ejemplo subsección}
	
	Hola buenos dias!!! \tjp{こんにちは!}
	
	% hbt! por si se quiere que aparezca en linea con el texto
	\begin{figure}[hbt!]
		\centering
		\includegraphics[width=0.5\textwidth]{ExampleFigures/ejemplo-diferencia-grafica-ni-kara-made-he.png}
		\caption{Diferencias entre \tjp{から・まで、に y へ}}
	\end{figure}
	
	\begin{table}[hbt!]
		\centering
		\begin{tabular}{| l | c | l |}
			\hline
			\textbf{\tjp{助詞}} & \textbf{Lectura} & \textbf{Uso principal} \\
			\hline
			\tjp{は} & Wa & Indica el tema principal de la frase \\
			\tjp{が} & Wa & Indica el sujeto de la frase \\
			\tjp{を} & O & Indica el objeto directo de la acción (quien la recibe) \\
			\tjp{に} & Wa & Indica direccion, tiempo, lugar. (muchas veces le sujeto indirecto) \\
			\hline
		\end{tabular}
		\caption{Varias particulas (\tjp{助詞})}
	\end{table}
	
	\grpre{
		Sustantivo+だ \\ Verbo informal \\ Adjetivo-な
	} {
		例えば
	}

	\chapter{Ejemplo titulo}

	Esta es una explicacion de ejemplo de algun concepto gramatical del \tjp{日本語}.

	\begin{defrapida}[Prueba de definicion 例えば]
		Esto es una prueba 例えば
	\end{defrapida}
	
	\begin{defrapida}
		Esto es una prueba 例えば
	\end{defrapida}

	\textbf{EJEMPLOS: }

	\jpex{
		私は吉良吉影です。
	}{
		Soy Kira yoshikage.
	}
	
	\jpex{
		私は吉良吉影です。
	}{
		Soy Kira yoshikage.
	}
	
	\begin{table}
		\centering
		\begin{tabular}{| l | c | l |}
			\hline
			\textbf{\tjp{助詞}} & \textbf{Lectura} & \textbf{Uso principal} \\
			\hline
			\tjp{は} & Wa & Indica el tema principal de la frase \\
			\tjp{が} & Wa & Indica el sujeto de la frase \\
			\tjp{を} & O & Indica el objeto directo de la acción (quien la recibe) \\
			\tjp{に} & Wa & Indica direccion, tiempo, lugar. (muchas veces le sujeto indirecto) \\
			\hline
		\end{tabular}
		\caption{Varias particulas segundo ejemplo (\tjp{助詞})}
	\end{table}
	
	\begin{figure}
		\centering
		\includegraphics[width=0.5\textwidth]{ExampleFigures/ejemplo-diferencia-grafica-ni-kara-made-he.png}
		\caption{Diferencias entre \tjp{から・まで、に y へ} segunda imagen.}
	\end{figure}

\end{document}

